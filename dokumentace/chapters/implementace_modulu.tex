%CHAPTER
\chapter{Implementace modulu}
%SECTION
\section{Generátor}
Generátor by měl být při vytváření PDF dokumentu rychlý, vykreslit co nejpřesněji prvky webového formuláře do vygenerovaného dokumentu a nebýt implementačně náročný.  
\subsection{TCPDF x mPDF}
%vyhody nevyhody, výsledný názor, omezení, chyby knihovny ((popis doimplementování našeho kódu)), 
Při analyzování dostupných PHP knihoven pro generování PDF souborů byly zjištěny 2 vyhovující knihovny, které můžou potencionálně splňovat potřebnou funkcionalitu, bohužel pouze 1 může být použit do vyvíjeného modulu. Po vytvoření jednoduchého souboru obsahující základní formulářové prvky bylo rozhodnuto, že knihovna \textbf{mPDF} bude použita pro generování PDF souborů. Důvody této volby jsou popsány níže.
\par
Jeden z důležitých faktorů lze označit skoro kompletní podporu \textit{CSS3} (Cascading Style Sheets 3) u \textbf{mPDF}, díky čemuž lze dosáhnout perfektního nastavení stylů pro jednotlivé objekty v dokumentu, zatímco \textbf{TCPDF} nepodporuje značné množství CSS parametrů (například parametr určující šířku vnějšího okraje prvku) a pro dosažení obdobného výsledku je zapotřebí značné množství jiných parametrů definující styl prvku.
\par
Důležitým faktorem při generování PDF je rychlost generování a paměťová náročnost. V tabulce \ref{tab:table_generators} lze vidět porovnání knihoven pro 2  PDF soubory, kdy komplexní PDF obsahovalo hlavně CSS styly, zatímco v dlouhém PDF byla vytvořena tabulka s více jak tisíci záznamy.
\begin{table}[h!]
\centering
\begin{tabular}{|l|l|l|l|l|} 
\hline
\textbf{Název} & \multicolumn{2}{l|}{\textbf{Komplexní PDF}} & \multicolumn{2}{l|}{\textbf{Dlouhé PDF}}  \\ 
\hline
               & \textbf{Paměť [MB]} & \textbf{Čas [ms]}     & \textbf{Paměť [MB]} & \textbf{Čas [ms]}   \\ 
\hline
TCPDF          & 74                  & 35944                 & 2,3                 & 96350               \\ 
\hline
mPDF           & 14                  & 11316                 & 22,5                & 4120                \\
\hline
\end{tabular}
\caption{Tabulka časové náročnosti a využité paměti při generování}
\label{tab:table_generators}
\end{table}
\par
Posledním a zároveň rozhodujícím důležitým faktorem je psaní PHP kódu pro vykreslování obsahu, kdy při psaní kódu u \textbf{mPDF} se využívá minimum funkcí pro nastavení parametrů PDF souboru (jako jsou například metadata), zatímco veškeré zobrazené elementy a text jsou psány v HTML stylu, který je snadno manipulovatelný a lze měnit parametry jednotlivých elementů (hodnota této vlastnosti bude oceněna hlavně u parseru). U \textbf{TCPDF} se zobrazovaný obsah vkládá pomocí předem vytvořených funkcí, kdy v některých případech tyto funkce obsahují mnoho parametrů, které si uživatel jen tak nezapamatuje a vždy bude potřebovat patřičnou dokumentaci pro správné použití (to bude zabírat mnoho času při vyvíjení nových modulů).
\par
Na závěr průzkumu lze říci, že ve většině případů je vhodné využít pro generování PDF souborů knihovnu \textbf{mPDF}. Pokud by ale uživatel potřeboval vygenerovat dokument ve stylu knihy (nulové využití CSS stylů a potřeba kvalitního vysázení textu), pak je lepší využít knihovnu \textbf{TCPDF}. 
%SECTION
\section{Parser}
%omezení, chyby knihovny, dosáhnutí výsledků (popis doimplementování našeho kódu)
%SECTION
\section{Implementované třídy}
%popis důležitých funkcí
%SECTION
\section{Výsledný vzhled PDF formuláře}
 %


