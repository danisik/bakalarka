\DeclarePairedDelimiter\ceil{\lceil}{\rceil}
\DeclarePairedDelimiter\floor{\lfloor}{\rfloor}

%CHAPTER
\chapter{Implementace modulu}
Při implementaci modulu bylo potřeba vybrat nejvhodnější knihovnu pro generování PDF souboru, kterou následně implementovat i s už předem vybraným parserem. Pro snadnější implementaci byly vytvořeny nové třídy pro ulehčení práce jak knihovnám, tak i budoucím vývojářům.
\section{Adresářová struktura modulu}
%popsat všechny složky, hlavní soubor atp.
Adresářová struktura modulu na serveru vypadá následovně:
\begin{itemize}
	\item \textbf{config} - Adresář obsahující konfigurační soubor \textit{configuration.xml}, ve kterém jsou uloženy často měněná data (rok konference, upozorňující info aj.).  
	\item \textbf{css} -  Adresář obsahující soubor \textit{style.css} s kaskádovými styly pro konfiguraci vzhledu HTML kódu a výsledného dokumentu. 
	\item \textbf{img} - Adresář obsahující obrázky použité v dokumentu.
	\item \textbf{lib} - Adresář obsahující zdrojové kódy knihoven třetích stran (generátor a parser).
	\item \textbf{src} - Adresář obsahující zdrojové kódy vytvořené autorem bakalářské práce.
	\item \textbf{how-to.txt} - Informační soubor, ve kterém se nachází implementační postupy (pro případné změny nebo rozšíření stávajícího modulu).
	\item \textbf{orlib.php} - Hlavní soubor modulu obsahující všechny 3 hlavní funkce modulu.
\end{itemize}

%SECTION
\section{Generátor}
Generátor by měl být při vytváření PDF dokumentu rychlý, vykreslit co nejpřesněji prvky webového formuláře do vygenerovaného dokumentu a nebýt implementačně náročný.  
%SUBSECTION
\subsection{TCPDF x mPDF}
Při analyzování dostupných PHP knihoven pro generování PDF souborů byly zjištěny 2 vyhovující knihovny, které můžou potencionálně splňovat potřebnou funkcionalitu, bohužel pouze 1 může být použit do vyvíjeného modulu. Po vytvoření jednoduchého souboru obsahující základní formulářové prvky bylo rozhodnuto, že knihovna \textbf{mPDF} bude použita pro generování PDF souborů. Důvody této volby jsou popsány níže.
\par
Jeden z důležitých faktorů lze označit skoro kompletní podporu \textit{CSS3} (Cascading Style Sheets 3) u \textbf{mPDF}, díky čemuž lze dosáhnout perfektního nastavení stylů pro jednotlivé objekty v dokumentu, zatímco \textbf{TCPDF} nepodporuje značné množství CSS parametrů (například parametr určující šířku vnějšího okraje prvku) a pro dosažení obdobného výsledku je zapotřebí značné množství jiných parametrů definující styl prvku.
\par
Důležitým faktorem při generování PDF je rychlost generování a paměťová náročnost. V tabulce \ref{tab:table_generators} lze vidět porovnání knihoven pro 2  PDF soubory, kdy komplexní PDF obsahovalo hlavně CSS styly, zatímco v dlouhém PDF byla vytvořena tabulka s více jak tisíci záznamy.
\begin{table}[h!]
\centering
\begin{tabular}{|l|l|l|l|l|} 
\hline
\textbf{Název} & \multicolumn{2}{l|}{\textbf{Komplexní PDF}} & \multicolumn{2}{l|}{\textbf{Dlouhé PDF}}  \\ 
\hline
               & \textbf{Paměť [MB]} & \textbf{Čas [ms]}     & \textbf{Paměť [MB]} & \textbf{Čas [ms]}   \\ 
\hline
TCPDF (v6.2.13)          & 74                  & 35944                 & 2,3                 & 96350               \\ 
\hline
mPDF   (v7.1.6)           & 14                  & 11316                 & 22,5                & 4120                \\
\hline
\end{tabular}
\caption{Tabulka časové náročnosti a využité paměti při generování}
\label{tab:table_generators}
\end{table}
\par
Posledním a zároveň rozhodujícím důležitým faktorem je psaní PHP kódu pro vykreslování obsahu, kdy při psaní kódu u \textbf{mPDF} se využívá minimum funkcí pro nastavení parametrů PDF souboru (jako jsou například metadata), zatímco veškeré zobrazené elementy a text jsou psány v HTML stylu, který je snadno manipulovatelný a lze měnit parametry jednotlivých elementů (hodnota této vlastnosti bude oceněna hlavně u parseru). U \textbf{TCPDF} se zobrazovaný obsah vkládá pomocí předem vytvořených funkcí, kdy v některých případech tyto funkce obsahují mnoho parametrů, které si uživatel jen tak nezapamatuje a vždy bude potřebovat patřičnou dokumentaci pro správné použití (to bude zabírat mnoho času při vyvíjení nových modulů).
\par
Na závěr průzkumu lze říci, že ve většině případů je vhodné využít pro generování PDF souborů knihovnu \textbf{mPDF}. Pokud by ale uživatel potřeboval vygenerovat dokument ve stylu knihy (nulové využití CSS stylů a potřeba kvalitního vysázení textu), pak je lepší využít knihovnu \textbf{TCPDF}. 
%SUBSECTION
\subsection{Popis vytvoření dokumentu}
%POPSAT POSTUP VYTVOŘENÍ JEDNOTLIVÝCH ČÁSTÍ, ŽE SE ČÁST VĚCÍ BERE Z KONFIGURÁKU, STYLY ZASE Z CSS SOUBORU, ZMÍNIT ÚPRAVY V KÓDU KNIHOVNY MPDF

%SECTION
\section{Parser}
%omezení, chyby knihovny, dosáhnutí výsledků (popis doimplementování našeho kódu)

%SUBSECTION
\subsection{Popis zpracování dokumentu}
%POPSAT POSTUP VYTVOŘENÍ JEDNOTLIVÝCH ČÁSTÍ, ZMÍNIT ÚPRAVY V KÓDU KNIHOVNY PDF PARSER

%SECTION
\section{Implementované třídy}
Pro modul bylo vytvořeno 7 tříd rozděleny do 4 souborů, které zajišťují veškerou pomocnou funkcionalitu (vytváření HTML kódu, konstanty aj.) při vytváření PDF souboru. 
%SUBSECTION
\subsection{Výčtové třídy}
\textbf{Výčtový typ} (neboli \textbf{Enum}) je datový typ určený pro uložení konstant programu, kdy každé z této konstanty je přiřazena jedna instance výčtu. Ve vytvářeném modulu byly použity 4 třídy jako výčtové typy. 
\par
Třída \textbf{Instruction} uchovává konstanty využité při generování titulku a informací ohledně vyplňování formuláře. Tyto konstanty reprezentují celkem 4 části dokumentu (Záhlaví, Titulek dokumentu, Jméno posuzovatele a Instrukční text pro vyplňování formuláře).
\par
Třída \textbf{FormElements} slouží pouze pro rozlišení použitých objektů na základní prvky formuláře. Zde byly použity prvky \textit{Radiobutton} a \textit{Textové pole}.
\par
Ve třídě \textbf{TextareaInfo} jsou uloženy veškeré konstanty pro hodnotící parametry, které jsou reprezentovány jako \textit{Textové pole} a pomocné funkce. Každý hodnotící parametr je zde reprezentován třemi konstantami (jednoznačný identifikátor, název a jeho popis). Dále se tu vyskytují 2 konstanty využité při vytváření formulářového prvku pomocí HTML kódu a následně i při jejich zpracovávání. Byla zde vytvořena i funkce \textit{getNotNeededConstants()} pro získání nepovinných hodnotících parametrů.
\par
Třída \textbf{RadiobuttonInfo} je téměř totožná s třídou \textbf{TextareaInfo} s tím rozdílem, že hodnotící parametry jsou reprezentovány jako \textit{Radiobutton}.
%SUBSECTION
\subsection{HTMLElements}
Hlavním důvodem vzniku této třídy byla snaha nevytvářet HTML kód přímo v hlavní funkci \textit{generate\_offline\_review\_form}, ale za použití nově vytvořených metod. Pomocí implementovaných metod lze vytvářet textová pole, radiobuttony a textové části dokumentu.
%SUBSECTION
\subsection{TextConversioner}
V některých případech je název vědeckého příspěvku nebo jméno posuzovatele příliš dlouhé, kdy narušuje vzhled výsledného dokumentu (narušení může nastat i při chybě programátora, pokud by byl instrukční text příliš dlouhý). Proto byla vytvořena třída \textbf{TextConversioner}, která má za úkol nejdříve zkontrolovat předaný text a porovnat ho se stanovenými konstantami určující maximální délku textu. Pokud je text delší než stanovená délka, tak se následně vypočte potřebný font pro vykreslení celého textu pomocí vzorce \eqref{eq:font_size} která se porovnává se stanovenými konstantami určující minimální font. 
\begin{equation}
newFontSize = \floor*{(fontSize *  maxTextLength) / textLength)} \label{eq:font_size}
\end{equation}

Za okolností že vypočtený font je menší než předem stanovený minimální font, je text zkrácen na velikost vypočtenou pomocí vzorce  \eqref{eq:text_length} a doplněn třemi tečkami na jeho konci. 

\begin{equation}
newLength = \floor*{(oldFontSize / minFontSize) / textLength} \label{eq:text_length}
\end{equation}
%SUBSECTION
\subsection{OwnXmlReader}
Třída načítající veškerý obsah konfiguračního souboru, který následně uloží do svých proměnných. Data uložená v konfiguračním souboru slouží pro nastavení textu vodoznaku a jako informační text pro uživatele.
\section{Výsledný vzhled PDF formuláře}


