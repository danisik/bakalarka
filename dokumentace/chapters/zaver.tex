%CHAPTER
\chapter{Závěr}
 V~rámci této práce se autor seznámil s~konferenčním systémem TSD a prostudoval již existující PHP knihovny pro práci s~formátem PDF.
\par
Na základě nabytých znalostí byl navržen a implementován modul umožňující vygenerování hodnotícího PDF souboru a jeho zpracování. Modul využívá dvě PHP knihovny třetích stran, které zajišťují veškerou funkcionalitu. 
\par
Pro generování hodnotícího PDF souboru byla nejdříve použita knihovna mPDF, která ale měla jeden zásadní nedostatek, který spočíval v~nezpůsobilosti sloučit PDF soubory s~verzí PDF nad 1.4. Po zjištění tohoto nedostatku byla vybrána knihovna TCPDI, se kterou lze tyto PDF soubory slučovat, ač lepší vzhled výsledného vygenerovaného souboru bylo lepší v~případě mPDF. 
\par
Pro zpracování hodnotícího PDF souboru byla použita knihovna PDF Parser, která měla za úkol zpracovat nahraný soubor a následně z~něj extrahovat hodnotící parametry. Protože tato knihovna pouze zpracovává veškerá data a nerozlišuje zda je daný objekt formulářový prvek, obrázek nebo text, musela být upravena. Upravení spočívalo v~přidání kontroly objektu při zpracovávání, zda se jedná o~daný hodnotící parametr. PDF Parser je distribuován pod licencí GPLv2, která umožňuje takovéto úpravy zdrojového kódu.
\par
Tady bude rekapitulace hodnocení z~testování
\par
Zadaní práce bylo bezpodmínečně splněno, výsledný modul umožňuje posuzovatelům nechat si vygenerovat hodnotící PDF soubor pro posuzování vědeckého příspěvku off-line a následně ho nahrát zpět na webový portál konference TSD. Veškerá funkcionalita byla zajištěna pomocí PHP knihoven třetích stran, které autor práce zakomponoval do systému s~využitím pomocného zdrojového kódu.