%CHAPTER
\chapter{Závěr}
 V~rámci této práce se autor seznámil s~konferenčním systémem TSD a prostudoval již existující PHP knihovny pro práci s~formátem PDF.
\par
Na základě nabytých znalostí byl navržen a implementován modul umožňující vygenerování hodnotícího souboru PDF a jeho zpracování. Modul využívá dvě PHP knihovny třetích stran, které zajišťují veškerou funkcionalitu. 
\par
Pro generování hodnotícího souboru PDF byla nejdříve použita knihovna mPDF, která ale měla jeden zásadní nedostatek, který spočíval v~nezpůsobilosti sloučit soubory PDF ve~verzi PDF nad 1.4. Po zjištění tohoto nedostatku byla vybrána knihovna TCPDI, se kterou lze tyto soubory PDF slučovat i přes to, že vzhled výsledného vygenerovaného souboru byl horší.
\par
Pro zpracování hodnotícího souboru PDF byla použita knihovna PDF Parser, která měla za úkol zpracovat nahraný soubor a následně z~něj extrahovat hodnotící parametry. Protože tato knihovna pouze zpracovává veškerá data a nerozlišuje, zda je daný objekt formulářový prvek, obrázek nebo text, musela být upravena. Úprava spočívala v~přidání kontroly objektu při zpracovávání, zda se jedná o~daný hodnotící parametr. PDF Parser je distribuován pod licencí GPLv2, která umožňuje takovéto úpravy zdrojového kódu.
\par
Zadaní práce bylo splněno ve všech bodech. Výsledný modul umožňuje recenzentům nechat si vygenerovat hodnotící soubor PDF pro posuzování vědeckého příspěvku off-line a následně ho nahrát zpět na webový portál konference TSD. Veškerá funkcionalita byla zajištěna pomocí PHP knihoven třetích stran, které autor práce zakomponoval do systému s~využitím pomocného zdrojového kódu.