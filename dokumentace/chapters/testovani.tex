%CHAPTER
\chapter{Ověření kvality software}
Po vytvoření modulu je nutné ověřit kvalitu řešení. Testování bylo zaměřeno hlavně na kvalitu vygenerovaného PDF dokumentu a jeho následného zpracování. Pro tento účel byl vytvořen testovací scénář pro posuzovatele vědeckých příspěvků. Důležitým faktorem při testování vytvořeného modulu je otestovat kompatibilitu s~nejvíce využívanými webovými a PDF prohlížeči. Vyplněné testovací scénáře se nachází v~příloze bakalářské práce.

%SECTION
\section{Testování modulu}

%SUBSECTION
\subsection{Generování PDF souboru}
Pro generování a stažení dokumentu bylo použito šest nejčastěji využívaných webových prohlížečů.
\par
Jako první testované prohlížeče lze zmínit \textit{Microsoft Edge (v42.17134.1.0)} a \textit{Internet Explorer (v11)} firmy Microsoft, které jsou již předinstalované na všech operačních systémech Windows od verze 10. Stažení proběhlo zcela v~pořádku, při stahování je potřeba zvolit možnost stažení souboru do počítače, nikoliv pouze otevření souboru. Vygenerovaný PDF soubor obsahoval všechny potřebné části pro hodnocení vědeckého příspěvku. Formulář bylo možné editovat.
\par
Generování bylo testováno i na webových prohlížečích \textit{Mozilla Firefox (v66.0.2)} a \textit{Google Chrome (v73.0.3683.103)}. Testování probíhalo na dvou operačních systémech Windows a Linux. I~zde proběhlo generování a stažení dokumentu bez sebemenších problémů, kdy Google Chrome ihned po stažení otevřel výchozí prohlížeč PDF souborů, zatímco Mozilla Firefox nabídla možnost stažení či pouze otevření PDF souboru. Stejně jako tomu bylo u~prvních dvou webových prohlížečů, bylo nutno soubor stáhnout, nikoliv pouze otevřít. Vygenerovaný soubor obsahoval editovatelný formulář i vědecký příspěvek, který je posuzován.
\par
Poslední dva webové prohlížece, na kterých bylo generování PDF souboru testováno, jsou \textit{Safari (v5.1.7)} a \textit{Opera (v58)}. Testování probíhalo pouze na systému Windows. Po stažení byl PDF soubor ihned otevřen výchozím PDF prohlížečem. Formulář byl editovatelný, veškeré části hodnotícího PDF souboru zde byly přítomny.
\par
Protože chytré telefony má dnes skoro každý, bylo proto nutné otestovat generování PDF souboru i na operačních systémech \textit{Android (v4.2.1 -- Jelly Bean)} a \textit{iOS (v12.2)}. Bez sebemenších problémů byl hodnotící PDF soubor vygenerován. Hodnotící formulář byl editovatelný.

%SUBSECTION
\subsection{Vyplnění a zpracování PDF souboru}
Pro vyplnění vygenerovaného hodnotícího PDF souboru bylo použito sedm PDF prohlížečů, viz \ref{tab:table_zpracovani}. Pro nahrání vyplněného PDF souboru byl použit webový prohlížeč Google Chrome.

\begin{table}[h!]
\centering
\begin{tabular}{|l|l|l|l|} 
\hline
\textbf{PDF prohlížeč} & \textbf{Verze PDF} & \textbf{Interaktivní formulář} & \textbf{Uložení hodnot}  \\ 
\hline
Adobe Acrobat Reader   & 1.6.0                & Ano                            & Ano                      \\ 
\hline
Adobe Acrobat Pro      & 1.6.0                & Ano                            & Ano                      \\ 
\hline
Evince                 & 1.4.0                & Ano                            & Ano                      \\ 
\hline
PDFElements            & 1.4.0                & Ano                            & Ano                      \\ 
\hline
Nitro Pro              & 1.4.0                & Ano                            & Ano                      \\ 
\hline
Sumatra                & X                  & Ne                             & Ne                       \\ 
\hline
Foxit - Linux          & 1.4.0                & Ano                            & Ano                      \\ 
\hline
Evince - Linux         & 1.4.0                & Ano                            & Ano                      \\
\hline
\end{tabular}
\caption{PDF prohlížeče použité při testování}
\label{tab:table_zpracovani}
\end{table}

\par
Nejčastěji využívané PDF prohlížeče jsou \textit{Adobe Acrobat Reader} a \textit{Adobe Acrobat Pro} ve verzi v19.01.020091. Hodnotící formulář zde byl editovatelný, obrázky se vyskytují v~perfektní kvalitě a při ukládání byla použita verze PDF 1.6.0. Po nahrání hodnotícího PDF souboru do webového portálu konference TSD byly všechny vyplněné hodnoty extrahování a úspěšně uloženy do databáze.
\par
Mezi známější PDF prohlížeče lze zařadit i \textit{Evince (v3.32)}, který byl použit pro testování na operačních systémech \textit{Windows 10} a \textit{Linux}. Vzhled formulářových prvků zde vypadal trochu jinak než jak tomu bylo u~Adobe prohlížečů, při ukládání PDF souboru byla použita verze PDF 1.4.0. Nahrání souboru a extrahování potřebných hodnot proběhlo bez sebemenších problémů na obou operačních systémech.
\par
Mezi méně známé PDF prohlížeče lze zařadit \textit{PDFElements (v6)}, \textit{Nitro Pro (v12)} a \textit{Foxit (v9.4.1.16828)}. Všechny tyto prohlížeče bezproblémově uložili všechny vyplněné hodnoty ve verzi PDF 1.4.0. Stejně jako tomu bylo u~\textit{Evince}, i zde se nevyskytl žádný problém při nahrávání souboru a extrakci požadovaných hodnot.
\par
Jako poslední PDF prohlížeč byl použit prohlížeč \textit{Sumatra (v3.1.2)}. Sumatra nepodporuje editovatelné formuláře, slouží pouze k~prohlížení PDF souborů. Proto z~tohoto důvodu nelze použít prohlížeč Sumatra pro vygenerovaný hodnotící PDF soubor.

%SUBSECTION
\subsection{Nalezené chyby při testování}
%že se neextrahovala data stejně jako když byla použta MPDF -> ta kontrola na value 'Off', chyba s~nahráváním (kdy se do systému neúspěšně nahrálo PDF
%dále zmínit že se po každém nahrání nepřepsali hodnoty, pokud v~novém formuláři něco nebylo vyplněno
Při testování modulu se objevily dvě větší chyby, které nebyly zapříčiněné použitou knihovnou.Jednalo se o~chyby při zpracovávání nahraného PDF souboru.
\par
První chyba byla objevena při testování knihovny TCPDI. Při kontrole extrahovaných hodnot z~nahraného PDF souboru se u~každé skupiny přepínacích tlačítek kontroluje, zda obsahuje hodnotu \uv{Off}. Pokud obsahuje, pak uživatel nezvolil ani jednu možnost v~dané skupině přepínacích tlačítek. Bohužel tento postup fungoval pouze u~formulářů vygenerovaných knihovnou mPDF, nikoliv u~formulářů vygenerovaných knihovnou TCPDI. U~TCPDI se při nezvolené možnosti ve skupině přepínacích tlačítek se daný hodnotící prvek ani neextrahuje, tudíž není dostupný a nelze ho nijak zkontrolovat. Proto byla přidána kontrola, zda extrahovaná data obsahují všechny hodnotící parametry u~všech formulářových prvků.
\par
Druhá chyba byla nalezena při ukládání hodnot do databáze. V~hodnotícím formuláři se vyskytují parametry, které uživatel nemusí vyplňovat. Pokud tyto položky nejsou vyplněné, nejsou při nahrávání extrahovány a následně uloženy do databáze, tudíž při nahrání nového PDF souboru se stávalo, že zde zůstali z~předchozího hodnocení komentáře. Proto bylo nutné zjistit zda tyto nepovinné parametry jsou vyplněny a pokud ne, byly explicitně vytvořeny v~modulu s~přiřazenou hodnotou, která byla prázdný řetězec. Tímto způsobem byly přepsány všechny parametry z~minulého hodnocení.

%SECTION
\section{Testovací scénář}
Testovací scénář je orientován především na vzhled celého dokumentu a rozpoložení jeho jednotlivých částí. Dále byly položeny otázky na bezproblémové ovládání modulu (vygenerování a nahrání PDF dokumentu). Poslední otázka byla zaměřena na poznámky a připomínky ohledně vytvořeného modulu, které by do budoucna mohli posloužit při případném rozšiřování modulu.
Výsledný testovací scénář předložený uživatelům konferenčního systému TSD obsahuje tyto otázky:
\begin{itemize}
	\item \textbf{Vyskytly se nějaké problémy při stahování či nahrávání PDF dokumentu?}
	\item \textbf{Jak by jste ohodnotil vzhled dokumentu jako celek?}
	\item \textbf{Jak hodnotíte vzhled formuláře a použité prvky reprezentující jednotlivé hodnotící parametry?}
	\item \textbf{Byla velikost textového pole dostatečně velká pro případně komentáře ohledně vědeckého příspěvku?}
	\item \textbf{Jak hodnotíte rychlost stažení a nahrání PDF dokumentu?} 
	\item \textbf{Byly Vámi vyplněné hodnoty správně nahrány do webového portálu?}
	\item \textbf{Děkuji za vyplnění dotazníku. Pokud máte jakékoliv další poznámky, připomínky či návrhy, uveďte je prosím zde:}
\end{itemize}

%SECTION
\section{Výsledky uživatelského testování}
\ref{chap:testovaci_reporty}
%SUBSECTION
\subsection{Vzhled PDF dokumentu}
%SUBSECTION
\subsection{Nalezené chyby a připomínky}

 