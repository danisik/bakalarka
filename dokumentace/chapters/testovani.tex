%CHAPTER
\chapter{Ověření kvality software}
Po vytvoření modulu je nutné ověřit kvalitu řešení. Testování bylo zaměřeno hlavně na kvalitu vygenerovaného PDF dokumentu a jeho následného zpracování. Pro tento účel byl vytvořen testovací scénář pro posuzovatele vědeckých příspěvků. Důležitým faktorem při testování vytvořeného modulu je otestovat kompatibilitu s~nejvíce využívanými webovými a PDF prohlížeči.Vyplněné testovací scénáře se nachází v~příloze bakalářské práce.

%SECTION
\section{Testovací scénář pro uživatele}
Testovací scénář je orientován především na vzhled celého dokumentu a rozpoložení jeho jednotlivých částí. Dále byly položeny otázky na bezproblémové ovládání modulu (vygenerování a nahrání PDF dokumentu). Poslední otázka byla zaměřena na poznámky a připomínky ohledně vytvořeného modulu, které by do budoucna mohli posloužit při případném rozšiřování modulu.
Výsledný testovací scénář předložený uživatelům konferenčního systému TSD obsahuje tyto otázky:
\begin{itemize}
	\item \textbf{Vyskytly se nějaké problémy při stahování či nahrávání PDF dokumentu?}
	\item \textbf{Jak by jste ohodnotil vzhled dokumentu jako celek?}
	\item \textbf{Jak hodnotíte vzhled formuláře a použité prvky reprezentující jednotlivé hodnotící parametry?}
	\item \textbf{Byla velikost textového pole dostatečně velká pro případně komentáře ohledně vědeckého příspěvku?}
	\item \textbf{Jak hodnotíte rychlost stažení a nahrání PDF dokumentu?} 
	\item \textbf{Byly Vámi vyplněné hodnoty správně nahrány do webového portálu?}
	\item \textbf{Děkuji za vyplnění dotazníku. Pokud máte jakékoliv další poznámky, připomínky či návrhy, uveďte je prosím zde:}
\end{itemize}

%SECTION
\section{Generování a stažení PDF souboru}
%TESTOVÁNÍ NA WEBOVÝCH PROHLÍŽEČÍCH
Scénář: doba generování, výsledek generování, Microsoft Edge, Internet Explorer, Mozilla Firefox, Google Chrome, Safari, Opera

%SECTION
\section{Vyplnění a nahrání PDF souboru}
%TESTOVÁNÍ V PDF PROHLÍŽEČÍCH
Jaká data byla použita při vyplňování, odkaz na tabulku, rychlost nahrání, správné výsledky zda byly extrahovány

\begin{table}[h!]
\centering
\begin{tabular}{|l|l|l|l|} 
\hline
\textbf{PDF prohlížeč} & \textbf{Verze PDF} & \textbf{Interaktivní formulář} & \textbf{Uložení hodnot}  \\ 
\hline
Adobe Acrobat Reader   & 1.6                & Ano                            & Ano                      \\ 
\hline
Adobe Acrobat Pro      & 1.6                & Ano                            & Ano                      \\ 
\hline
Evince                 & 1.4                & Ano                            & Ano                      \\ 
\hline
PDFElements            & 1.4                & Ano                            & Ano                      \\ 
\hline
Nitro Pro              & 1.4                & Ano                            & Ano                      \\ 
\hline
Sumatra                & X                  & Ne                             & Ne                       \\ 
\hline
Foxit - Linux          & 1.4                & Ano                            & Ano                      \\ 
\hline
Evince - Linux         & 1.4                & Ano                            & Ano                      \\
\hline
\end{tabular}
\end{table}

%SECTION
\section{Nalezené chyby při testování}
že se neextrahovala data stejně jako když byla použta MPDF -> ta kontrola na value 'Off', chyba s~nahráváním (kdy se do systému neúspěšně nahrálo PDF

%SECTION
\section{Výsledky uživatelského testování}
\subsection{Vzhled PDF dokumentu}
\subsection{Nalezené chyby a připomínky}
 