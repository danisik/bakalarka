\chapter{Formát PDF}
Formát \textbf{PDF} (\textbf{P}ortable \textbf{D}ocument \textbf{F}ormat) je souborový formát vyvinutý společností Adobe v roce 1992. PDF formát byl vyvinut za účelem konzistentní prezentace dokumentů (spustitelné na více zařízeních a různých platformách). Díky konzistenci lze dosáhnout toho, že PDF soubor vytvořený a uložený v systému Windows bude zobrazen totožně na systémech Mac, na všech distribucích Linuxu nezávisle na použitém PDF prohlížeči (Adobe Reader, Foxit a další).
\par 
V PDF souboru lze uchovávat velice širokou škálu dat, včetně formátovaného textu, vektorové grafiky a rastrových obrazů, nebo například informace o rozložení, velikosti a tvaru stránky. Informace definující umístění jednotlivých položek (jsou zde zahrnuty i editovací objekty pro formuláře) na stránce jsou zde uloženy též. Do dokumentu lze ukládat i metadata. Metadata jsou informace uložené v hlavičce souboru a lze do nich uložit název dokumentu, autora dokumentu, předmět a klíčová slova. Je zde možnost uložit heslo, aby byl dokument přístupný pouze autorizovaným uživatelům. Všechny tyto informace jsou uloženy ve standardním formátu \cite{PDFTechTerms, PDFWhatIs}.

\section{Komprese dat v PDF}
PDF soubory moho být poměrně kompaktní, o mnoho menší než ekvivalentní postscriptové soubory. Tato vlastnost je dosažena nejen lepší strukturou dat, ale i díky kompresním algoritmům, které jsou velice efektivní. Typ komprese dat PDF souboru lze zjistit pomocí textového editoru, který dokáže zpracovat binární data, vyhledáním klíčového slova \textbf{/Filter}. Níže jsou popsány kompresní algoritmy využívané v PDF \cite{PDFPrepressure}.
\begin{itemize}
	\item \textbf{CCITT G3/G4} - Algoritmus je bezeztrátový a využívá se pro vykreslení černobílých obrázků.
	\item \textbf{JPEG} - JPEG algoritmus může být jak ztrátový, tak i bezeztrátový. V Acrobatu se využívá pouze ztrátový s 5 stupni komprese. Využívá se pro barevné a šedotónové obrázky.
	\item \textbf{JPEG2000} - Rychlejší algoritmus na bázi JPEGu. Víceméně se nepoužívá, jelikož není kompatibilní se staršímy systémy a vysokýmy nároky na procesor.
	\item \textbf{Flate} - Bezeztrátový algoritmus, vychází z kompresních algoritmů LZ77 a Huffmanova kódování.
	\item \textbf{JBIG2} - Alternativní k CCITT. V Dnešní době se nevyužívá z důvodu pomalejší komprese než je u jeho protějšku.
	\item \textbf{LZW} - Komprimací LZW algoritmem lze dosáhnout až o polovinu menší velikosti díky komprimaci veškerého textu a operátorů v souboru.
	\item \textbf{RLE} - Bezeztrátový algoritmus pro vykreslování černobílých obrázků. Nahrazen efektivnějším algoritmem CCITT.
	\item \textbf{ZIP} - Bezeztrátový algoritmus, učinější než jeho protějšek LZW.
\end{itemize}

\section{Vnitřní struktura PDF}
Vnitřní reprezentace PDF souboru je rozdělena na sekce, které jsou znázorněny na obrázku \ref{fig:pdf_internal_structure}.

\begin{figure}[h!]
\centering
\includegraphics[width=4cm]{img/pdf_internal_structure}
\caption{Interní struktura PDF souboru}
\label{fig:pdf_internal_structure}
\end{figure}

Z obrázku lze vyčíst, že se zde vyskytují 4 hlavní sekce: \textit{Header, Body, Cross-reference a trailer}. Díky jedné z vlastností PDF formátu se při úpravě souboru staré sekce neodstraní, místo toho se pouze na jeho konci vytvoří nové sekce \cite{PDFInfoSec}.
\begin{itemize}
	\item \textbf{Header} - Hlavička souboru je uložena na první řádce, obsahující primárně použitou verzi PDF.
	\begin{figure}[h!]
	\centering
	\includegraphics[width=12cm]{img/pdf_hlavicka}
	\caption{Hlavička PDF souboru}
	\label{fig:pdf_header}
	\end{figure}
	
	\item \textbf{Body} - V těle dokumentu jsou uložena veškerá data objektů reprezentující celý dokument. Objekty jsou referencovány v tabulce Cross-reference z důvodu rozprostření částí dat patřících k danému objektu po celé sekci. Pomocí objektů lze reprezentovat například obrázky, text a zvukové nahrávky.
	\begin{figure}[h!]
	\centering
	\includegraphics[width=12cm]{img/pdf_body}
	\caption{Data jednoho objektu}
	\label{fig:pdf_body}
	\end{figure}
	\item \textbf{Cross-reference table} - Jinak nazývána \textbf{xref} je tabulka obsahující reference na veškeré objekty uložené v těle a v kódu začíná řetězcem \textit{xref}. Reference uložená v tabulce je reprezentována na 2 řádcích pomocí řetězce a skládá se z 5 částí o celkové velikosti 20 bytů včetně oddělovačů \textit{CRLF}:
	\begin{itemize}
		\item \textit{Číslo objektu} - Jednoznačný číselný identifikátor objektu.
		\item \textit{Počet subobjektů} - Počet částí daného objektu vyskytujícího se v dokumentu.
		\item \textit{Začátek objektu} - Tvoří většinu řetězce (prvních 10 bytů) a určuje offset od začátku PDF dokumentu až po začátek daného objektu.
		\item \textit{Generační číslo objektu} - Vyjadřuje jak často byl objekt vymazán při úpravě dokumentu. 
		\item \textit{Identifikátor využití} - Nabývá hodnot \textit{f} (free) nebo \textit{n} (use) a vyjadřuje, zda je objekt vyobrazen v dokumentu.
	\end{itemize}
	\newpage
	\begin{figure}[h!]
	\centering
	\includegraphics[width=12cm]{img/pdf_xref}
	\caption{Ukázka jednoduché xref tabulky}
	\label{fig:pdf_xref}
	\end{figure}
	\item \textbf{Trailer} -  Trailer je seznam informací, ze kterých lze snadno zjistit umístění xref tabulky
	\begin{itemize}
		\item \textit{Size} - 
		\item \textit{Prev} - 
		\item \textit{Root} -
		\item \textit{Encrypt} - 
		\item \textit{Info} - 
		\item \textit{ID} -
		\item \textit{XrefStm} - 

	\end{itemize}
	\begin{figure}[h!]
	\centering
	\includegraphics[width=12cm]{img/pdf_trailer}
	\caption{Ukázka traileru}
	\label{fig:pdf_trailer}
	\end{figure}
\end{itemize}