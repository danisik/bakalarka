%CHAPTER
\chapter{Úvod}
TSD (\textbf{T}ext, \textbf{S}peech and \textbf{D}ialogue) je konference zabývající se problémy zpracování přirozeného jazyka. Mezi nejčastěji probíraná témata se řadí například rozpoznávání řeči, modelování řeči,textové korpusy, značkování textu a mnoho dalších. Konference se koná každý rok v září a místo konání se střídá mezi Brnem (pořadatelem je Fakulta informatiky Masarykovy Univerzity) a Plzní (pořadatelem je Fakulta aplikovaných věd Západočeské univerzity v Plzni). Tento rok bude konference organizována právě Západočeskou univerzitou, a poprvé se bude konat za hranicemi České Republiky, přesněji na Slovinsku ve městě Ljubljana.
\par
 Ke každé konferenci existuje webový portál vytvořený daným pořadatelem, na nějž jsou od uživatelů nahrávány vědecké příspěvky. Tyto příspěvky jsou poté hodnoceny recenzenty (převážně členy programového výboru) formou online formuláře a na základě konečného hodnocení jednotlivých parametrů a na doporučení recenzentů jsou tyto příspěvky schváleny organizátorem a mohou být prezentovány na konferenci, nebo jsou zamítnuty z důvodu nedostatečného hodnocení. Modul vytvářený autorem bude implementován do webového portálu organizovaný Fakultou aplikovaných věd.
\par
Cílem této práce je prostudovat strukturu PDF formátu, který je pro vytváření editovacích formulářů nejvhodnější a byl vybrán zadávajícím jako standard, tak i funkcionalitu volně dostupných PHP knihoven pro generování a parsování PDF souborů obsahujících editovatelný formulář, aby existovala možnost ohodnocení daného vědeckého příspěvku i v místech, kde není dostupné internetové připojení, neboli off-line. Tento PDF soubor musí obsahovat hodnotící formulář se všemi hodnotícími parametry, text vědeckého příspěvku doplněný o vodoznak. Pro generování a parsování musí být použity výhradně knihovny v jazyce PHP, jelikož není vhodné využívat aplikace třetích stran spustitelné z terminálu. Modul musí být nezávislý na platformě a lze ho upravovat v jakémkoliv PDF prohlížeči nezávisle na verzi PDF. Před vytvořením modulu na testovací verzi webového portálu bude potřeba projít zdrojové soubory webového portálu pro seznámení s již existujícími funkcionalitami a zařadit do portálu i náš modul. Z dřívějších let je zde naimplementován totožný modul pro generování a parsování PDF souborů, bohužel tento modul nesplňuje veškeré body zadání právě z důvodu použití nevhodného parseru.
 