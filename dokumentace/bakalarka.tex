%%%%%%%%%%%%%%%%%%%%%%%%%%%%%%%%%%%%%%%%%%%%%%%%%%%%%%%%%%
%
% Vzor pro sazbu kvalifikační práce
%
% Západočeská univerzita v Plzni
% Fakulta aplikovaných věd
% Katedra informatiky a výpočetní techniky
%
% Petr Lobaz, lobaz@kiv.zcu.cz, 2016/03/14
%
%%%%%%%%%%%%%%%%%%%%%%%%%%%%%%%%%%%%%%%%%%%%%%%%%%%%%%%%%%

% Možné jazyky práce: czech, english
% Možné typy práce: BP (bakalářská), DP (diplomová)
\documentclass[czech,BP]{thesiskiv}

% Definujte údaje pro vstupní strany
%
% Jméno a příjmení; kvůli textu prohlášení určete, 
% zda jde o mužské, nebo ženské jméno.
\author{Vojtěch Danišík}
\declarationmale

%alternativa: 
%\declarationfemale

% Název práce
\title{Generátor a parser formulářů recenzí příspěvků na konferenci TSD}

% 
% Texty abstraktů (anglicky, česky)
% 

%The text of the abstract (in English). It contains the English translation of the thesis title and a short description of the thesis.
\abstracttexten{Generator and Parser of Submission Review Forms for the TSD Conference.
\newline The goal of this thesis is to create PHP module, which will be easily integrate into existing information system for managing TSD conference. First part of the thesis explains standard PDF format and forms created in PDF. Subsequently, there are described existing PHP libraries for generating off-line PDF forms of scientific contribution, their advantages and disadvantages. Second part of the thesis describing existing PHP libraries for parsing PDF file. The module was tested by conference system users and multiple PDF browsers were used. Test results are part of this thesis.}


%Text abstraktu (česky). Obsahuje krátkou anotaci (cca 10 řádek) v češtině. Budete ji potřebovat i při vyplňování údajů o bakalářské práci ve STAGu. Český i anglický abstrakt by měly být na stejné stránce a měly by si obsahem co možná nejvíce odpovídat (samozřejmě není možný doslovný překlad!).
\abstracttextcz{Cílem bakalářské práce je vytvořit jednoduše integrovatelný PHP modul do již existujícího informačního systému pro správu konference TSD. První část práce důkladně vysvětluje standardní formát PDF a formuláře vytvořené v PDF. Následně jsou popsány existující PHP knihovny pro generování off-line PDF formuláře daného vědeckého příspěvku, jejich výhody a nevýhody. Druhá část práce popisuje existující PHP knihovny pro parsování souborů ve formátu PDF. Modul byl otestován uživateli konferenčního systému a bylo použito více PDF prohlížečů. Výsledky testování jsou součástí této práce.}

% Na titulní stranu a do textu prohlášení se automaticky vkládá 
% aktuální rok, resp. datum. Můžete je změnit:
%\titlepageyear{2016}
%\declarationdate{1. března 2016}

% Ve zvláštních případech je možné ovlivnit i ostatní texty:
%
%\university{Západočeská univerzita v Plzni}
%\faculty{Fakulta aplikovaných věd}
%\department{Katedra informatiky a výpočetní techniky}
%\subject{Bakalářská práce}
%\titlepagetown{Plzeň}
%\declarationtown{Plzni}

%%%%%%%%%%%%%%%%%%%%%%%%%%%%%%%%%%%%%%%%%%%%%%%%%%%%%%%%%%
%
% DODATEČNÉ BALÍČKY PRO SAZBU
% Jejich užívání či neužívání záleží na libovůli autora 
% práce
%
%%%%%%%%%%%%%%%%%%%%%%%%%%%%%%%%%%%%%%%%%%%%%%%%%%%%%%%%%%

% Zařadit literaturu do obsahu
\usepackage[nottoc,notlot,notlof]{tocbibind}

% Umožňuje vkládání obrázků
\usepackage[pdftex]{graphicx}

% Odkazy v PDF jsou aktivní; navíc se automaticky vkládá
% balíček 'url', který umožňuje např. dělení slov
% uvnitř URL
\usepackage[pdftex]{hyperref}
\hypersetup{colorlinks=true,
  unicode=true,
  linkcolor=black,
  citecolor=black,
  urlcolor=black,
  bookmarksopen=true}

% Při používání citačního stylu csplainnatkiv
% (odvozen z csplainnat, http://repo.or.cz/w/csplainnat.git)
% lze snadno modifikovat vzhled citací v textu
\usepackage[numbers,sort&compress]{natbib}

\usepackage{svg}
\usepackage{amsmath}

%%%%%%%%%%%%%%%%%%%%%%%%%%%%%%%%%%%%%%%%%%%%%%%%%%%%%%%%%%
%
% VLASTNÍ TEXT PRÁCE
%
%%%%%%%%%%%%%%%%%%%%%%%%%%%%%%%%%%%%%%%%%%%%%%%%%%%%%%%%%%
\begin{document}
%
\maketitle
\tableofcontents

\thispagestyle{empty}
\setcounter{page}{0} 
\chapter{Úvod}
TSD (\textbf{T}ext, \textbf{S}peech and \textbf{D}ialogue) je konference zabývající se problémy zpracování přirozeného jazyka. Mezi nejčastěji probíraná témata se řadí: rozpoznávání řeči, modelování řeči,textové korpusy, značkování textu a mnoho dalších. Konference se koná každý rok v září a místo konání se střídá mezi Brnem (pořadatelem je Fakulta informatiky Masarykovy Univerzity) a Plzní (pořadatelem je Fakulta aplikovaných věd Západočeské univerzity v Plzni). Tento rok bude konference organizována právě Západočeskou univerzitou, a poprvé se bude konat za hranicemi České Republiky, přesněji na Slovinsku ve městě Ljubljana.
\par
 Ke každé konferenci existuje webový portál vytvořený daným pořadatelem, na nějž jsou od uživatelů nahrávány vědecké příspěvky. Tyto příspěvky jsou poté hodnoceny recenzenty (převážně členy programového výboru) formou online formuláře a na základě konečného hodnocení jednotlivých parametrů a na doporučení recenzentů jsou tyto příspěvky schváleny organizátorem a mohou být prezentovány na konferenci, nebo jsou zamítnuty z důvodu nedostatečného hodnocení. Modul vytvářený autorem bude implementován do webového portálu organizovaný Fakultou aplikovaných věd.
\par
Cílem této práce je prostudovat strukturu PDF formátu, který je pro vytváření editovacích formulářů nejvhodnější a byl vybrán zadávajícím jako standard, tak i funkcionalitu volně dostupných PHP knihoven pro generování a parsování PDF souborů obsahujících editovatelný formulář, aby existovala možnost ohodnocení daného vědeckého příspěvku i v místech, kde není dostupné internetové připojení, neboli off-line. Tento PDF soubor musí obsahovat hodnotící formulář se všemi hodnotícími parametry, text vědeckého příspěvku doplněný o vodoznak. Pro generování a parsování musí být použity výhradně knihovny v jazyce PHP, jelikož není vhodné využívat aplikace třetích stran spustitelné z terminálu. Modul musí být nezávislý na platformě a lze ho upravovat v jakémkoliv PDF prohlížeči nezávisle na verzi PDF. Před vytvořením modulu na testovací verzi webového portálu bude potřeba projít zdrojové soubory webového portálu pro seznámení s již existujícími funkcionalitami a zařadit do portálu i náš modul. Z dřívějších let je zde naimplementován totožný modul pro generování a parsování PDF souborů, bohužel tento modul nesplňuje veškeré body zadání právě z důvodu použití nevhodného parseru.
 
\chapter{Formát PDF}
Formát \textbf{PDF} (\textbf{P}ortable \textbf{D}ocument \textbf{F}ormat) je souborový formát vyvinutý společností Adobe v roce 1992. PDF formát byl vyvinut za účelem konzistentní prezentace dokumentů (spustitelné na více zařízeních a různých platformách). Díky konzistenci lze dosáhnout toho, že PDF soubor vytvořený a uložený v systému Windows bude zobrazen totožně na systémech Mac, na všech distribucích Linuxu nezávisle na použitém PDF prohlížeči (Adobe Reader, Foxit a další).
\par 
V PDF souboru lze uchovávat velice širokou škálu dat, včetně formátovaného textu, vektorové grafiky a rastrových obrazů, nebo například informace o rozložení, velikosti a tvaru stránky. Informace definující umístění jednotlivých položek (jsou zde zahrnuty i editovací objekty pro formuláře) na stránce jsou zde uloženy též. Do dokumentu lze ukládat i metadata. Metadata jsou informace uložené v hlavičce souboru a lze do nich uložit název dokumentu, autora dokumentu, předmět a klíčová slova. Je zde možnost uložit heslo, aby byl dokument přístupný pouze autorizovaným uživatelům. Všechny tyto informace jsou uloženy ve standardním formátu \cite{PDFTechTerms, PDFWhatIs}.

\section{Komprese dat v PDF}
PDF soubory moho být poměrně kompaktní, o mnoho menší než ekvivalentní postscriptové soubory. Tato vlastnost je dosažena nejen lepší strukturou dat, ale i díky kompresním algoritmům, které jsou velice efektivní. Typ komprese dat PDF souboru lze zjistit pomocí textového editoru, který dokáže zpracovat binární data, vyhledáním klíčového slova \textbf{/Filter}. Níže jsou popsány kompresní algoritmy využívané v PDF \cite{PDFPrepressure}.
\begin{itemize}
	\item CCITT G3/G4 - Algoritmus je bezeztrátový a využívá se pro vykreslení černobílých obrázků.
	\item JPEG - JPEG algoritmus může být jak ztrátový, tak i bezeztrátový. V Acrobatu se využívá pouze ztrátový s 5 stupni komprese. Využívá se pro barevné a šedotónové obrázky.
	\item JPEG2000 - Rychlejší algoritmus na bázi JPEGu. Víceméně se nepoužívá, jelikož není kompatibilní se staršímy systémy a vysokýmy nároky na procesor.
	\item Flate - Bezeztrátový algoritmus, vychází z kompresních algoritmů LZ77 a Huffmanova kódování.
	\item JBIG2 - Alternativní k CCITT. V Dnešní době se nevyužívá z důvodu pomalejší komprese než je u jeho protějšku.
	\item LZW - Komprimací LZW algoritmem lze dosáhnout až o polovinu menší velikosti díky komprimaci veškerého textu a operátorů v souboru.
	\item RLE - Bezeztrátový algoritmus pro vykreslování černobílých obrázků. Nahrazen efektivnějším algoritmem CCITT.
	\item ZIP - Bezeztrátový algoritmus, učinější než jeho protějšek LZW.
\end{itemize}

%CHAPTER
\chapter{Výsledky testování modulu}
 
%CHAPTER
\chapter{Závěr}
 V~rámci této práce se autor seznámil s~konferenčním systémem TSD a prostudoval již existující PHP knihovny pro práci s~formátem PDF.
\par
Na základě nabytých znalostí byl navržen a implementován modul umožňující vygenerování hodnotícího PDF souboru a jeho zpracování. Modul využívá dvě PHP knihovny třetích stran, které zajišťují veškerou funkcionalitu. 
\par
Pro generování hodnotícího PDF souboru byla nejdříve použita knihovna mPDF, která ale měla jeden zásadní nedostatek, který spočíval v~nezpůsobilosti sloučit PDF soubory s~verzí PDF nad 1.4. Po zjištění tohoto nedostatku byla vybrána knihovna TCPDI, se kterou lze tyto PDF soubory slučovat, ač lepší vzhled výsledného vygenerovaného souboru bylo lepší v~případě mPDF. 
\par
Pro zpracování hodnotícího PDF souboru byla použita knihovna PDF Parser, která měla za úkol zpracovat nahraný soubor a následně z~něj extrahovat hodnotící parametry. Protože tato knihovna pouze zpracovává veškerá data a nerozlišuje zda je daný objekt formulářový prvek, obrázek nebo text, musela být upravena. Upravení spočívalo v~přidání kontroly objektu při zpracovávání, zda se jedná o~daný hodnotící parametr. PDF Parser je distribuován pod licencí GPLv2, která umožňuje takovéto úpravy zdrojového kódu.
\par
Tady bude rekapitulace hodnocení z~testování
\par
Zadaní práce bylo bezpodmínečně splněno, výsledný modul umožňuje posuzovatelům nechat si vygenerovat hodnotící PDF soubor pro posuzování vědeckého příspěvku off-line a následně ho nahrát zpět na webový portál konference TSD. Veškerá funkcionalita byla zajištěna pomocí PHP knihoven třetích stran, které autor práce zakomponoval do systému s~využitím pomocného zdrojového kódu.

% 
% PRO ANGLICKOU SAZBU JE NUTNÉ ZMĚNIT
% CITAČNÍ STYL!
%
\bibliographystyle{csplainnatkiv}
{\raggedright\small
\bibliography{literature/literatura}
}

\end{document}
