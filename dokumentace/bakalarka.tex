%%%%%%%%%%%%%%%%%%%%%%%%%%%%%%%%%%%%%%%%%%%%%%%%%%%%%%%%%%
%
% Vzor pro sazbu kvalifikační práce
%
% Západočeská univerzita v Plzni
% Fakulta aplikovaných věd
% Katedra informatiky a výpočetní techniky
%
% Petr Lobaz, lobaz@kiv.zcu.cz, 2016/03/14
%
%%%%%%%%%%%%%%%%%%%%%%%%%%%%%%%%%%%%%%%%%%%%%%%%%%%%%%%%%%

% Možné jazyky práce: czech, english
% Možné typy práce: BP (bakalářská), DP (diplomová)
\documentclass[czech,BP]{thesiskiv}

% Definujte údaje pro vstupní strany
%
% Jméno a příjmení; kvůli textu prohlášení určete, 
% zda jde o mužské, nebo ženské jméno.
\author{Vojtěch Danišík}
\declarationmale

%alternativa: 
%\declarationfemale

% Název práce
\title{Generátor a parser formulářů recenzí příspěvků na konferenci TSD}

% 
% Texty abstraktů (anglicky, česky)
% 

%The text of the abstract (in English). It contains the English translation of the thesis title and a short description of the thesis.
\abstracttexten{Generator and Parser of Submission Review Forms for the TSD Conference.
The goal of this thesis is to create PHP module, which will be easily integrate into existing information system for managing TSD conference. First part of the thesis explains standard PDF format and forms created in PDF. Subsequently, there are described existing PHP libraries for generating off-line PDF forms of scientific contribution, their advantages and disadvantages. Second part of the thesis describing existing PHP libraries for parsing PDF file. The module was tested by conference system users and multiple PDF browsers were used. Test results are part of this thesis. }


%Text abstraktu (česky). Obsahuje krátkou anotaci (cca 10 řádek) v češtině. Budete ji potřebovat i při vyplňování údajů o bakalářské práci ve STAGu. Český i anglický abstrakt by měly být na stejné stránce a měly by si obsahem co možná nejvíce odpovídat (samozřejmě není možný doslovný překlad!).
\abstracttextcz{
%Cílem bakalářské práce je vytvořit jednoduše integrovatelný modul do již existujícího informačního systému pro správu konference TSD. První část modulu má za úkol generování off-line formuláře daného vědeckého příspěvku, který byl přiřazen hodnotiteli. Formulář musí být vygenerován ve standardním formátu (v modulu je použit PDF Forms). Druhá část modulu má za úkol nahrát do systému vyplněný hodnotící formulář a následně z něj extrahovat potřebná data pro uložení do databáze konferenčního systému TSD.

Cílem bakalářské práce je vytvořit jednoduše integrovatelný PHP modul do již existujícího informačního systému pro správu konference TSD. První část práce důkladně vysvětluje standardní formát PDF a formuláře vytvořené v PDF. Následně jsou popsány existující PHP knihovny pro generování off-line PDF formuláře daného vědeckého příspěvku, jejich výhody a nevýhody. Druhá část práce popisuje existující PHP knihovny pro parsování souborů ve formátu PDF. Modul byl otestován uživateli konferenčního systému a bylo použito více PDF prohlížečů. Výsledky testování jsou součástí této práce.

%Cílem práce je vytvořit modul - jednoduše integrovatelný do existujícího informačního systému pro správu konference TSD, který bude generovat formuláře pro recenzenty jednotlivých vědeckých příspěvků. Formuláře již v systému existují v on-line podobě, tj. recenzent vyplní formulář na webu konference. Nyní chceme tento systém rozšířit o možnost vyplňování formuláře off-line: Formulář musí být vygenerovaný v nějakém standardním formátu (např. PDF Forms, apod.), recenzent ho vyplní a následně uploaduje do systému, kde je opět třeba tento formulář zpracovat, extrahovat z něj vyplněné údaje a převést je do interních struktur informačního systému. Vygenerování formuláře a jeho následné zpracování po vyplnění recenzentem je předmětem této BP.
}

% Na titulní stranu a do textu prohlášení se automaticky vkládá 
% aktuální rok, resp. datum. Můžete je změnit:
%\titlepageyear{2016}
%\declarationdate{1. března 2016}

% Ve zvláštních případech je možné ovlivnit i ostatní texty:
%
%\university{Západočeská univerzita v Plzni}
%\faculty{Fakulta aplikovaných věd}
%\department{Katedra informatiky a výpočetní techniky}
%\subject{Bakalářská práce}
%\titlepagetown{Plzeň}
%\declarationtown{Plzni}

%%%%%%%%%%%%%%%%%%%%%%%%%%%%%%%%%%%%%%%%%%%%%%%%%%%%%%%%%%
%
% DODATEČNÉ BALÍČKY PRO SAZBU
% Jejich užívání či neužívání záleží na libovůli autora 
% práce
%
%%%%%%%%%%%%%%%%%%%%%%%%%%%%%%%%%%%%%%%%%%%%%%%%%%%%%%%%%%

% Zařadit literaturu do obsahu
\usepackage[nottoc,notlot,notlof]{tocbibind}

% Umožňuje vkládání obrázků
\usepackage[pdftex]{graphicx}

% Odkazy v PDF jsou aktivní; navíc se automaticky vkládá
% balíček 'url', který umožňuje např. dělení slov
% uvnitř URL
\usepackage[pdftex]{hyperref}
\hypersetup{colorlinks=true,
  unicode=true,
  linkcolor=black,
  citecolor=black,
  urlcolor=black,
  bookmarksopen=true}

% Při používání citačního stylu csplainnatkiv
% (odvozen z csplainnat, http://repo.or.cz/w/csplainnat.git)
% lze snadno modifikovat vzhled citací v textu
\usepackage[numbers,sort&compress]{natbib}

%%%%%%%%%%%%%%%%%%%%%%%%%%%%%%%%%%%%%%%%%%%%%%%%%%%%%%%%%%
%
% VLASTNÍ TEXT PRÁCE
%
%%%%%%%%%%%%%%%%%%%%%%%%%%%%%%%%%%%%%%%%%%%%%%%%%%%%%%%%%%
\begin{document}
%
\maketitle
\tableofcontents

%CHAPTER
\chapter{Úvod}
TSD (\textbf{T}ext, \textbf{S}peech and \textbf{D}ialogue) je mezinárodní konference zabývající se například problémy zpracování, překladu a rozpoznávání přirozeného jazyka nebo analýzou řeči. Mezi nejčastěji probíraná témata se řadí například rozpoznávání řeči, modelování řeči,textové korpusy, značkování textu a mnoho dalších. Konference se koná každý rok v~září a místo konání se střídá mezi Brnem (pořadatelem je Fakulta informatiky Masarykovy Univerzity) a Plzní (pořadatelem je Fakulta aplikovaných věd Západočeské univerzity v~Plzni). Tento rok bude konference organizována právě Západočeskou univerzitou, a poprvé se bude konat za hranicemi České Republiky, přesněji ve Slovinsku ve městě Ljubljana.
\par
Ke konferenci existuje webový portál, na nějž jsou od uživatelů nahrávány vědecké příspěvky. Tyto příspěvky jsou poté hodnoceny recenzenty (převážně členy programového výboru) formou online formuláře a na základě konečného hodnocení jednotlivých parametrů a na doporučení recenzentů jsou tyto příspěvky schváleny organizátorem a mohou být prezentovány na konferenci. Modul, vytvářený autorem, bude implementován do webového portálu konference TSD.
\par
Cílem této práce je prostudovat strukturu PDF formátu, který je pro vytváření editovatelných formulářů nejvhodnější a byl vybrán zadávajícím jako standard, tak i funkcionalitu volně dostupných PHP knihoven pro generování a parsování PDF souborů obsahujících editovatelný formulář, aby existovala možnost ohodnocení daného vědeckého příspěvku i v~místech, kde není dostupné internetové připojení, neboli off-line. Vytvořený PDF soubor musí obsahovat hodnotící formulář se všemi hodnotícími parametry doplněný o~text vědeckého příspěvku. Pro generování a parsování musí být použity výhradně knihovny v~jazyce PHP, jelikož není vhodné využívat aplikace třetích stran spustitelné z~terminálu. Modul musí být nezávislý na platformě a lze ho upravovat v~jakémkoliv PDF prohlížeči. Před vytvořením modulu na testovací verzi webového portálu bude potřeba projít zdrojové soubory webového portálu pro seznámení s~již existujícími funkcionalitami a zařadit do portálu i náš modul. Z~dřívějších let je zde naimplementován totožný modul pro generování a parsování PDF souborů, bohužel tento modul nesplňuje veškeré body zadání právě z~důvodu použití nevhodného parseru.
 

% 
% PRO ANGLICKOU SAZBU JE NUTNÉ ZMĚNIT
% CITAČNÍ STYL!
%
\bibliographystyle{csplainnatkiv}
{\raggedright\small
\bibliography{literature/literatura}
}

\end{document}
