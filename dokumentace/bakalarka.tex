%%%%%%%%%%%%%%%%%%%%%%%%%%%%%%%%%%%%%%%%%%%%%%%%%%%%%%%%%%
%
% Vzor pro sazbu kvalifikační práce
%
% Západočeská univerzita v Plzni
% Fakulta aplikovaných věd
% Katedra informatiky a výpočetní techniky
%
% Petr Lobaz, lobaz@kiv.zcu.cz, 2016/03/14
%
%%%%%%%%%%%%%%%%%%%%%%%%%%%%%%%%%%%%%%%%%%%%%%%%%%%%%%%%%%

% Možné jazyky práce: czech, english
% Možné typy práce: BP (bakalářská), DP (diplomová)
\documentclass[czech,BP]{thesiskiv}

% Definujte údaje pro vstupní strany
%
% Jméno a příjmení; kvůli textu prohlášení určete, 
% zda jde o mužské, nebo ženské jméno.
\author{Vojtěch Danišík}
\declarationmale

%alternativa: 
%\declarationfemale

% Název práce
\title{Generátor a parser formulářů recenzí příspěvků na konferenci TSD}

% 
% Texty abstraktů (anglicky, česky)
% 

%The text of the abstract (in English). It contains the English translation of the thesis title and a short description of the thesis.
\abstracttexten{Generator and Parser of Submission Review Forms for the TSD Conference.
\newline The goal of this thesis is to create PHP module, which will be easily integrate into existing information system for managing TSD conference. First part of the thesis explains standard PDF format and forms created in PDF. Subsequently, there are described existing PHP libraries for generating off-line PDF forms of scientific contribution, their advantages and disadvantages. Second part of the thesis describing existing PHP libraries for parsing PDF file. The module was tested by conference system users and multiple PDF browsers were used. Test results are part of this thesis.}


%Text abstraktu (česky). Obsahuje krátkou anotaci (cca 10 řádek) v češtině. Budete ji potřebovat i při vyplňování údajů o bakalářské práci ve STAGu. Český i anglický abstrakt by měly být na stejné stránce a měly by si obsahem co možná nejvíce odpovídat (samozřejmě není možný doslovný překlad!).
\abstracttextcz{Cílem bakalářské práce je vytvořit jednoduše integrovatelný PHP modul do již existujícího informačního systému pro správu konference TSD. První část práce důkladně vysvětluje standardní formát PDF a formuláře vytvořené v PDF. Následně jsou popsány existující PHP knihovny pro generování off-line PDF formuláře daného vědeckého příspěvku, jejich výhody a nevýhody. Druhá část práce popisuje existující PHP knihovny pro parsování souborů ve formátu PDF. Modul byl otestován uživateli konferenčního systému a bylo použito více PDF prohlížečů. Výsledky testování jsou součástí této práce.}

% Na titulní stranu a do textu prohlášení se automaticky vkládá 
% aktuální rok, resp. datum. Můžete je změnit:
%\titlepageyear{2016}
%\declarationdate{1. března 2016}

% Ve zvláštních případech je možné ovlivnit i ostatní texty:
%
%\university{Západočeská univerzita v Plzni}
%\faculty{Fakulta aplikovaných věd}
%\department{Katedra informatiky a výpočetní techniky}
%\subject{Bakalářská práce}
%\titlepagetown{Plzeň}
%\declarationtown{Plzni}

%%%%%%%%%%%%%%%%%%%%%%%%%%%%%%%%%%%%%%%%%%%%%%%%%%%%%%%%%%
%
% DODATEČNÉ BALÍČKY PRO SAZBU
% Jejich užívání či neužívání záleží na libovůli autora 
% práce
%
%%%%%%%%%%%%%%%%%%%%%%%%%%%%%%%%%%%%%%%%%%%%%%%%%%%%%%%%%%

% Zařadit literaturu do obsahu
\usepackage[nottoc,notlot,notlof]{tocbibind}

% Umožňuje vkládání obrázků
\usepackage[pdftex]{graphicx}

% Odkazy v PDF jsou aktivní; navíc se automaticky vkládá
% balíček 'url', který umožňuje např. dělení slov
% uvnitř URL
\usepackage[pdftex]{hyperref}
\hypersetup{colorlinks=true,
  unicode=true,
  linkcolor=black,
  citecolor=black,
  urlcolor=black,
  bookmarksopen=true}

% Při používání citačního stylu csplainnatkiv
% (odvozen z csplainnat, http://repo.or.cz/w/csplainnat.git)
% lze snadno modifikovat vzhled citací v textu
\usepackage[numbers,sort&compress]{natbib}

\usepackage{svg}
\usepackage{amsmath}

%%%%%%%%%%%%%%%%%%%%%%%%%%%%%%%%%%%%%%%%%%%%%%%%%%%%%%%%%%
%
% VLASTNÍ TEXT PRÁCE
%
%%%%%%%%%%%%%%%%%%%%%%%%%%%%%%%%%%%%%%%%%%%%%%%%%%%%%%%%%%
\begin{document}
%
\maketitle
\tableofcontents

\chapter{Úvod}
TSD (\textbf{T}ext, \textbf{S}peech and \textbf{D}ialogue) je konference zabývající se problémy zpracování přirozeného jazyka. Mezi nejčastěji probíraná témata se řadí: rozpoznávání řeči, modelování řeči,textové korpusy, značkování textu a mnoho dalších. Konference se koná každý rok v září a místo konání se střídá mezi Brnem (pořadatelem je Fakulta informatiky Masarykovy Univerzity) a Plzní (pořadatelem je Fakulta aplikovaných věd Západočeské univerzity v Plzni). Tento rok bude konference organizována právě Západočeskou univerzitou, a poprvé se bude konat za hranicemi České Republiky, přesněji na Slovinsku ve městě Ljubljana.
\par
 Ke každé konferenci existuje webový portál vytvořený daným pořadatelem, na nějž jsou od uživatelů nahrávány vědecké příspěvky. Tyto příspěvky jsou poté hodnoceny recenzenty (převážně členy programového výboru) formou online formuláře a na základě konečného hodnocení jednotlivých parametrů a na doporučení recenzentů jsou tyto příspěvky schváleny organizátorem a mohou být prezentovány na konferenci, nebo jsou zamítnuty z důvodu nedostatečného hodnocení. Modul vytvářený autorem bude implementován do webového portálu organizovaný Fakultou aplikovaných věd.
\par
Cílem této práce je prostudovat strukturu PDF formátu, který je pro vytváření editovacích formulářů nejvhodnější a byl vybrán zadávajícím jako standard, tak i funkcionalitu volně dostupných PHP knihoven pro generování a parsování PDF souborů obsahujících editovatelný formulář, aby existovala možnost ohodnocení daného vědeckého příspěvku i v místech, kde není dostupné internetové připojení, neboli off-line. Tento PDF soubor musí obsahovat hodnotící formulář se všemi hodnotícími parametry, text vědeckého příspěvku doplněný o vodoznak. Pro generování a parsování musí být použity výhradně knihovny v jazyce PHP, jelikož není vhodné využívat aplikace třetích stran spustitelné z terminálu. Modul musí být nezávislý na platformě a lze ho upravovat v jakémkoliv PDF prohlížeči nezávisle na verzi PDF. Před vytvořením modulu na testovací verzi webového portálu bude potřeba projít zdrojové soubory webového portálu pro seznámení s již existujícími funkcionalitami a zařadit do portálu i náš modul. Z dřívějších let je zde naimplementován totožný modul pro generování a parsování PDF souborů, bohužel tento modul nesplňuje veškeré body zadání právě z důvodu použití nevhodného parseru.
 
\chapter{Formát PDF}
Formát \textbf{PDF} (\textbf{P}ortable \textbf{D}ocument \textbf{F}ormat) je souborový formát vyvinutý společností Adobe v roce 1992. PDF formát byl vyvinut za účelem konzistentní prezentace dokumentů (spustitelné na více zařízeních a různých platformách). Díky konzistenci lze dosáhnout toho, že PDF soubor vytvořený a uložený v systému Windows bude zobrazen totožně na systémech Mac, na všech distribucích Linuxu nezávisle na použitém PDF prohlížeči (Adobe Reader, Foxit a další).
\par 
V PDF souboru lze uchovávat velice širokou škálu dat, včetně formátovaného textu, vektorové grafiky a rastrových obrazů, nebo například informace o rozložení, velikosti a tvaru stránky. Informace definující umístění jednotlivých položek (jsou zde zahrnuty i editovací objekty pro formuláře) na stránce jsou zde uloženy též. Do dokumentu lze ukládat i metadata. Metadata jsou informace uložené v hlavičce souboru a lze do nich uložit název dokumentu, autora dokumentu, předmět a klíčová slova. Je zde možnost uložit heslo, aby byl dokument přístupný pouze autorizovaným uživatelům. Všechny tyto informace jsou uloženy ve standardním formátu \cite{PDFTechTerms, PDFWhatIs}.

\section{Komprese dat v PDF}
PDF soubory moho být poměrně kompaktní, o mnoho menší než ekvivalentní postscriptové soubory. Tato vlastnost je dosažena nejen lepší strukturou dat, ale i díky kompresním algoritmům, které jsou velice efektivní. Typ komprese dat PDF souboru lze zjistit pomocí textového editoru, který dokáže zpracovat binární data, vyhledáním klíčového slova \textbf{/Filter}. Níže jsou popsány kompresní algoritmy využívané v PDF \cite{PDFPrepressure}.
\begin{itemize}
	\item CCITT G3/G4 - Algoritmus je bezeztrátový a využívá se pro vykreslení černobílých obrázků.
	\item JPEG - JPEG algoritmus může být jak ztrátový, tak i bezeztrátový. V Acrobatu se využívá pouze ztrátový s 5 stupni komprese. Využívá se pro barevné a šedotónové obrázky.
	\item JPEG2000 - Rychlejší algoritmus na bázi JPEGu. Víceméně se nepoužívá, jelikož není kompatibilní se staršímy systémy a vysokýmy nároky na procesor.
	\item Flate - Bezeztrátový algoritmus, vychází z kompresních algoritmů LZ77 a Huffmanova kódování.
	\item JBIG2 - Alternativní k CCITT. V Dnešní době se nevyužívá z důvodu pomalejší komprese než je u jeho protějšku.
	\item LZW - Komprimací LZW algoritmem lze dosáhnout až o polovinu menší velikosti díky komprimaci veškerého textu a operátorů v souboru.
	\item RLE - Bezeztrátový algoritmus pro vykreslování černobílých obrázků. Nahrazen efektivnějším algoritmem CCITT.
	\item ZIP - Bezeztrátový algoritmus, učinější než jeho protějšek LZW.
\end{itemize}

%CHAPTER
\chapter{Ověření kvality software}
Po vytvoření modulu je nutné ověřit kvalitu řešení. Testování bylo zaměřeno hlavně na kvalitu vygenerovaného dokumentu PDF a jeho následné zpracování. Pro tento účel byl vytvořen testovací scénář pro recenzenta vědeckých příspěvků. Důležitým faktorem při testování vytvořeného modulu je otestovat kompatibilitu s~nejvíce využívanými webovými a prohlížeči PDF. Vyplněné testovací scénáře se nachází v~příloze bakalářské práce.

%SECTION
\section{Testování modulu}

%SUBSECTION
\subsection{Generování souboru PDF}
Pro generování a stažení dokumentu bylo použito šest nejčastěji využívaných webových prohlížečů.
\par
Jako první testované prohlížeče lze zmínit \textit{Microsoft Edge (v42.17134.1.0)} a \textit{Internet Explorer (v11)} firmy Microsoft, které jsou již předinstalované na všech operačních systémech Windows od verze 10. Stažení proběhlo zcela v~pořádku, při stahování je potřeba zvolit možnost stažení souboru do počítače, nikoliv pouze otevření souboru. Vygenerovaný soubor PDF obsahoval všechny potřebné části pro hodnocení vědeckého příspěvku. Formulář bylo možné editovat.
\par
Generování bylo testováno i na webových prohlížečích \textit{Mozilla Firefox (v66.0.2)} a \textit{Google Chrome (v73.0.3683.103)}. Testování probíhalo na dvou operačních systémech Windows 10 a Linux (Ubuntu v18.04.1 LTS, Kernel 4.15.0--44--generic. I~zde proběhlo generování a stažení dokumentu bez sebemenších problémů, kdy Google Chrome ihned po stažení otevřel výchozí prohlížeč souborů PDF, zatímco Mozilla Firefox nabídl možnost stažení či pouze otevření souboru PDF. Stejně jako tomu bylo u~prvních dvou webových prohlížečů, bylo nutno soubor stáhnout, nikoliv pouze otevřít. Vygenerovaný soubor obsahoval editovatelný formulář i vědecký příspěvek, který je posuzován.
\par
Poslední dva webové prohlížece, na kterých bylo generování souboru PDF testováno, jsou \textit{Safari (v5.1.7)} a \textit{Opera (v58)}. Testování probíhalo pouze na systému Windows. Po stažení byl soubor PDF ihned otevřen výchozím prohlížečem PDF. Formulář byl editovatelný, veškeré části hodnotícího souboru PDF zde byly přítomny.
\par
Protože chytré telefony má dnes skoro každý, bylo proto nutné otestovat generování souboru PDF i na operačních systémech \textit{Android (v4.2.1 -- Jelly Bean)} a \textit{iOS (v12.2)}. Bez sebemenších problémů byl hodnotící soubor PDF vygenerován. Hodnotící formulář byl editovatelný.

%SUBSECTION
\subsection{Vyplnění a zpracování souboru PDF}
Při vyplnění vygenerovaného hodnotícího souboru PDF bylo použito sedm prohlížečů PDF, viz \ref{tab:table_zpracovani}. K~nahrání vyplněného souboru PDF byl použit webový prohlížeč Google Chrome.

\begin{table}[h!]
\centering
\begin{tabular}{|l|l|l|l|} 
\hline
\textbf{Prohlížeč PDF} & \textbf{Verze PDF} & \textbf{Interaktivní formulář} & \textbf{Uložení hodnot}  \\ 
\hline
Adobe Acrobat Reader   & 1.6.0                & Ano                            & Ano                      \\ 
\hline
Adobe Acrobat Pro      & 1.6.0                & Ano                            & Ano                      \\ 
\hline
Evince                 & 1.4.0                & Ano                            & Ano                      \\ 
\hline
PDFElements            & 1.4.0                & Ano                            & Ano                      \\ 
\hline
Nitro Pro              & 1.4.0                & Ano                            & Ano                      \\ 
\hline
Sumatra                & X                  & Ne                             & Ne                       \\ 
\hline
Foxit - Linux          & 1.4.0                & Ano                            & Ano                      \\ 
\hline
Evince - Linux         & 1.4.0                & Ano                            & Ano                      \\
\hline
\end{tabular}
\caption{Prohlížeče PDF použité při testování}
\label{tab:table_zpracovani}
\end{table}

\par
Nejčastěji využívané prohlížeče PDF jsou \textit{Adobe Acrobat Reader} a \textit{Adobe Acrobat Pro} ve verzi v19.01.020091. Hodnotící formulář zde byl editovatelný, obrázky se vyskytují v~perfektní kvalitě a při ukládání byla použita verze PDF 1.6.0. Po nahrání souboru PDF v~recenzním řízení do webového portálu konference TSD byly všechny vyplněné hodnoty extrahovány a úspěšně uloženy do databáze.
\par
Mezi známější prohlížeče PDF lze zařadit i \textit{Evince (v3.32)}, který byl použit pro testování na operačních systémech \textit{Windows 10} a \textit{Linux}. Vzhled formulářových prvků zde vypadal trochu jinak než jak tomu bylo u~Adobe prohlížečů, při ukládání souboru PDF byla použita verze PDF 1.4.0. Nahrání souboru a extrahování potřebných hodnot proběhlo bez sebemenších problémů na obou operačních systémech.
\par
Mezi méně známé prohlížeče PDF lze zařadit \textit{PDFElements (v6)}, \textit{Nitro Pro (v12)} a \textit{Foxit (v9.4.1.16828)}. Všechny tyto prohlížeče bezproblémově uložily všechny vyplněné hodnoty ve verzi PDF 1.4.0. Stejně jako tomu bylo u~\textit{Evince}, i zde se nevyskytl žádný problém při nahrávání souboru a extrakci požadovaných hodnot.
\par
Jako poslední prohlížeč PDF byl použit prohlížeč \textit{Sumatra (v3.1.2)}. Sumatra nepodporuje editovatelné formuláře, slouží pouze k~prohlížení souborů PDF. Proto z~tohoto důvodu nelze použít prohlížeč Sumatra pro vygenerovaný soubor PDF v~recenzním řízení.

%SUBSECTION
\subsection{Nalezené chyby při testování}
\label{subsec:chyby_pri_testovani}
Při testování modulu se objevily dvě větší chyby, které nebyly zapříčiněné použitou knihovnou. Jednalo se o~chyby při zpracovávání nahraného souboru PDF.
\par
První chyba byla objevena při testování knihovny TCPDI. Při kontrole extrahovaných hodnot z~nahraného souboru PDF se u~každé skupiny přepínacích tlačítek kontroluje, zda obsahuje hodnotu \uv{Off}. Pokud obsahuje, pak uživatel nezvolil ani jednu možnost v~dané skupině přepínacích tlačítek. Bohužel tento postup fungoval pouze u~formulářů vygenerovaných knihovnou mPDF, nikoliv u~formulářů vygenerovaných knihovnou TCPDI. U~TCPDI se při nezvolené možnosti ve skupině přepínacích tlačítek se daný hodnotící prvek ani neextrahuje, tudíž není dostupný a nelze ho nijak zkontrolovat. Proto byla přidána kontrola, zda extrahovaná data obsahují všechny hodnotící parametry u~všech formulářových prvků.
\par
Druhá chyba byla nalezena při ukládání hodnot do databáze. V~hodnotícím formuláři se vyskytují parametry, které uživatel nemusí vyplňovat. Pokud tyto položky nejsou vyplněné, nejsou při nahrávání extrahovány a následně uloženy do databáze, tudíž při nahrání nového souboru PDF se stávalo, že zde zůstali z~předchozího hodnocení komentáře. Proto bylo nutné zjistit zda tyto nepovinné parametry jsou vyplněny a pokud ne, byly explicitně vytvořeny v~modulu s~přiřazenou hodnotou, která byla prázdný řetězec. Tímto způsobem byly přepsány všechny parametry z~minulého hodnocení.

%SECTION
\section{Testovací scénář}
Testovací scénář je orientován především na vzhled celého dokumentu a rozložení jeho jednotlivých částí. Dále byly položeny otázky na bezproblémové ovládání modulu (vygenerování a nahrání dokumentu PDF). Poslední otázka byla zaměřena na poznámky a připomínky ohledně vytvořeného modulu, které by do budoucna mohly posloužit při případném rozšiřování modulu.
Výsledný testovací scénář předložený uživatelům konferenčního systému TSD obsahuje tyto otázky:
\begin{itemize}
	\item \textbf{Vyskytly se nějaké problémy při stahování či nahrávání dokumentu PDF?}
	\item \textbf{Jak byste ohodnotil vzhled dokumentu jako celek?}
	\item \textbf{Jak hodnotíte vzhled formuláře a použité prvky reprezentující jednotlivé hodnotící parametry?}
	\item \textbf{Byla velikost textového pole dostatečně velká pro případné komentáře ohledně vědeckého příspěvku?}
	\item \textbf{Jak hodnotíte rychlost stažení a nahrání dokumentu PDF?} 
	\item \textbf{Byly Vámi vyplněné hodnoty správně nahrány do webového portálu?}
	\item \textbf{Děkuji za vyplnění dotazníku. Pokud máte jakékoliv další poznámky, připomínky či návrhy, uveďte je, prosím, zde:}
\end{itemize}

%SECTION
\section{Výsledky uživatelského testování}
Funkčnost modulu byla otestována dvěma uživateli webového portálu konference TSD. Níže jsou popsány souhrny jednotlivých částí testovacího reportu. V~kapitole \ref{chap:testovaci_reporty} jsou podrobně rozepsány všechny odpovědi uživatelu.
%SUBSECTION
\subsection{Vzhled dokumentu PDF}
Vzhled hodnotícího formuláře byl podle uživatelů vcelku přehledný a věcný, ale svým vzhledem je ani neurazil, ani nenadchnul. Velikost textových polí byla dostatečně velká pro zadání všech informací. Ohledně celkového vzhledu dokumentu měli uživatelé námitku na odsazení výběrových tlačítek. Ocenili by především větší odsazení skupin výběrových tlačítek od ostatních sekcí.
%SUBSECTION
\subsection{Nalezené chyby a připomínky}
Při stahování a nahrávání dokumentů se nevyskytly žádné problémy a vše proběhlo bez výraznějších prodlev mezi jedna až dvěmi vteřinami. Při nahrávání vyplněného souboru PDF v~recenzním řízení do webového portálu se u~prvního uživatele vyskytly problémy, zatímco druhý neměl žádné připomínky. Připomínky prvního uživatele byly: portál nezná háčky, čárky a opravuje je na nesmyslné znaky, je schopný nahrát i nesmyslný text ve stylu samých teček a problém při ukládání hodnot do databáze, který spočíval v~přepisování starých hodnot novými. Co se týče špatného zobrazování háčků a čárek tak tento problém byl vysvětlen v~kapitole \ref{subsec:nedostatky_v_mpdf}. Ohledně nesmyslného textu, většina hodnotících parametrů je povinná, a proto musí být i v~off-line formuláři náležitě vyplněny všechny povinné parametry. Poslední připomínka byla zjištěna ke konci autorova testování, kdy už byly rozeslány testovací reporty všem uživatelům, viz kapitola \ref{subsec:chyby_pri_testovani}.  

 
%CHAPTER
\chapter{Závěr}
 

% 
% PRO ANGLICKOU SAZBU JE NUTNÉ ZMĚNIT
% CITAČNÍ STYL!
%
\bibliographystyle{csplainnatkiv}
{\raggedright\small
\bibliography{literature/literatura}
}

\end{document}
